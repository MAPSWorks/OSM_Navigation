%podkapitoly
%\section{Protokol NMEA 0183}
%\section{Formát GPX}
%\section{GPS babel}
%\section{Gps multi tracker}

\chapter{Spracovanie dát z GPS prístroja}

\section{Protokol NMEA 0183}
\paragraph{NMEA}
"Štandardizovaný komunikačný protokol pre prenos dát z GPS do iného
elektronického zariadenia (najčastejšie PC) v reálnom čase. Dáta z GPS sú
tematický rozdelené do niekoľkých skupín - ASCII reťazcov a posielané von
z prístroja."[10]

Tento protokol na najčastejšie používa pre prenos GPS dát potrebných na zobrazenie pozície používateľa, ktorý smeruje z prístrojov do PC zariadení.
Každá veta začína znakom \$. Za znakom \$ pokračuje označenie dát, za ktorým sa nachádzajú konkrétne dáta. Dáta sú oddelené čiarkou, pre dobré rozonanie. V prípade že sa konkrétne dáta nedajú vypočítať (nedostupnosť signálu, nedostatok dát potrebných pre výpočet) sú prístrojom na danej pozícii posielané ako nuly ako prázdne dáta.  Niektoré prijímače neposielajú všetky dáta.

\paragraph{Popis NMEA prvkov}
Pred každý z kódov sa pridáva príznak GP\\
RMB  = Minimálne potrebné navigačné informácie\\
RMC  = Odporúčané minimálne tranzitné údaje \\
GGA  = Globálny pozičný systém Fixné dáta(zemepisné súradnice, nadmorská výška, počet satelitov a pod.)\\
VTG  = Kvalita a rýchlosť aktuálnej trate\\
RMA  = Minimálne potrebné navigačné informácie\\
%GSA  = mód GPS prijímača, SVs a DOP hodnoty\\
GSV  =  Viditeľné Satelity\\
\begin{center}
 \textbf{Príklad NMEA dát}
\begin{verbatim}
Dáta sú dostupné:
$GPGGA,110532.538,4910.1966,N,01852.6027,E,0,00,1.4,2.5,M,44.0,M,,00*57
$GPVTG,161.5,T,,M,043.8,N,081.1,K,N*06

Dáta nie sú dostupné:
$GPGGA,185837.507,0000.0000,N,00000.0000,E,0,00,50.0,0.0,M,0.0,M,,00*74
$GPRMC,185837.507,V,0000.0000,N,00000.0000,E,,,050309,,*19
\end{verbatim}
\end{center}

\section{Formát GPX}
\paragraph{}
Bol navrhnutý tak, že dáta sú uložené v XML formáte. Tento formát má veľa
výhod. GPX (z angl. GPS eXchange Format) je formát súborov pre ukladanie dát vo forme HTML tagov\footnote{\textbf{tag} - značka metadát}. Dáta sú tak vhodné do spracovania počítačom, ale zároveň sú jednoducho čitateľné aj pre človeka. Tento formát je určený hlavne na komunikáciu internetových webov. 

Forma GPX vychádza z XML\footnote{\textbf{XML} - rozšíriteľný značkovací jazyk (eXtensible Markup Language)} formátu. Tento formát sa vyvinul z hlavne html štandardu. Vznikol hlavne preto, lebo samotný html formát postupom času nabral na zložitosti a množstve príkazov. XML formát nemá predpísané značky (tagy), no zároveň jeho syntax je podstatne prísnejšia ako u HTML\footnote{\textbf{HTML} - Hypertextový značkový jazyk HyperText Markup Language)} dát.
\paragraph{}
\textbf{Existujú dve verzie tohto formátu a to:}
\begin{list}{•}
\item 1.0 používaná od roku 2002
\item
\item 1.1 používaná od 9. Augusta 2004 
\end{list}
\paragraph{}
\textbf{Verzie sú vzájomne kompatibilné. Verzia 1.1 priniesla nasledujúce zmeny:}
\begin{list}{•}
\item tagy s obecnými informáciami o celom súbore sa presunuli do tagu metadata
\item
\item miesto tagov url a urlname sa používa nový tag link
\item voliteľné rozširujúce tagy sa presunuli do tagu extension
\end{list}

Programy by mali jasne deklarovať s ktorou verziou GPX vedia pracovať, aby sa
predišlo situáciám kedy nebude možné niektoré GPX dáta spracovať kvôli
nekompatibilite verzie.

\subsection{Element gpx}

\begin{center}
\textbf{Tento element je koreňovým elementom GPX dokumentu}
\begin{verbatim}                                                    
"< gpx
   version = " 1.1 [1] "
   creator = " xsd:string [1] " >
       < metadata > metadataType </ metadata > [0..1]
       < wpt > wptType </ wpt > [0..*]
       < rte > rteType </ rte > [0..*]
       < trk > trkType </ trk > [0..*]
       < extensions > extensionsType </ extensions > [0..1]
</ gpx > "[2]
\end{verbatim}
\end{center}
Hodnoty v zátvorkách znázorňujú početnosť dát v zázname. Každý začiatočný tag musí mať aj svoju ukončovaciu značku.\\


\begin{center}
\textbf{Príklad GPX schémy typu waypoint}
\begin{verbatim}
<wpt lat="49.210526565" lon="18.758500053">
  <ele>435.097072</ele>
  <time>2009-10-29T21:50:20Z</time>
  <name>TP1149</name>
  <cmt>TP1149</cmt>
  <desc>TP1149</desc>
</wpt>
\end{verbatim}
\end{center}

Jednotlivé tagy môžu obsahovať niektoré z atribútov(pr: `lat`,`lon`). Samotné atribúty majú len názvy (žiadne začiatočné ani ukončovacie značky). Názvy tagov sú vždy na začiatku a na konci(pr: `ele`,`wpt`). Tagy musia obsahovať aj začiatočné aj ukončovacie značky.


\begin{center}
\textbf{Príklad GPX schémy typu track}
\begin{verbatim}
<...>
<name> xsd:string </name> [0..1] ?
<cmt> xsd:string </cmt> [0..1] ?
<desc> xsd:string </desc> [0..1] ?
<src> xsd:string </src> [0..1] ?
<link> linkType </link> [0..*] ?
<number> xsd:nonNegativeInteger </number> [0..1] ?
<type> xsd:string </type> [0..1] ?
<extensions> extensionsType </extensions> [0..1] ?
<trkseg> trksegType </trkseg> [0..*] ?
</...>
\end{verbatim}
\end{center}

V príklade sú dáta iba popisom(informáciou o trase. Popisy nám slúžia na identifikáciu trasy, pričom existencia jednotlivých  popisov je podmienená typom prístroja, ktorý dáta zapisuje. Samotné dáta trás potrebné na vykreslenie sa nachádzajú v časti \textit{trkseg} (názov trksegType v príklade).

\section{Skytraq datalogger}
\paragraph{}Program vyvíjaný Jesper Zedlitz-om, je schopný spracovať dáta z
prístroja s identickým názvom. Tieto dáta ukladá do výstupného formátu GPX.
Samotný vývoj softvéru prešiel pod program GPSbabel, no doposiaľ je iba v testovacej verzii. Tento program nemá GUI interface a spúšťanie prebieha iba v termináli. S programom je možné konfigurovať daný prístroj čo je dôležité pri zapisovaní(logovaní) trás do prístroja. Viac informácii o konfigurácii prístroja sa nachádza v tretej kapitole.
\begin{center}
 \textbf{Vzorové príkazy:}
\begin{verbatim}
skytraq-datalogger --info
skytraq-datalogger <OPTIONS> ACTION
skytraq-datalogger --dump > vystup.gpx 
- uloží výsledné dáta do súboru vystup.gpx
\end{verbatim}
\end{center}
\paragraph{}Všetky ostané informácie sa vypíšu po príkaze info. Používateľ si intuitívne dokáže zvoliť nastavenia prístroja.
Časť options je nepovinná(nastavenia prístroja), no v nasledujúcej časti zadávame žiadanú akciu(`--info`,`--dump`).

\section{GPS babel}
\paragraph{}GPSbabel konvertuje dáta medzi GPS prijímačmi a mapovými programami. Tiež je
vhodný na manipuláciu s týmito dátami.
Tento program umožňuje prácu s dátami priamo na úrovni hardvéru t.j. prácu
priamo s GPS prijímačmi. Podpora dátových formátov stále stúpa a je dostupný pod
rôznymi operačnými systémami.
Nekonvertuje ani neposiela mapy. Pracuje len s dátami ktoré sú na mape uložené,
dátami ako trasami a zaujímavými bodmi. 
Pre túto prácu ho bolo vhodné použiť práve preto, že je dostupný pod rôznymi platformami\footnote{\textbf{platforma} = operačný systém}.
Komunikácia s prístrojom používaným v tejto práci - Sky\_traq data
logger bola nedávno implementovaná. Predchádzajúci softvér postačoval
na nastavenie komunikácie sťahovanie dát, no nebolo s ním možné nastaviť cieľový
bod potrebný na navigáciu s prís\-tro\-jom.
Aby sme mohli použiť tento program museli sme nainštalovať beta verziu programu,
pretože stabilná verzia doposiaľ, nepodporovala tento prístroj.

\subsection{Príkazy v programe GPSbabel}
\paragraph{}Keďže s programom GPSbabel sa pracuje na úrovni príkazového riadku príkazy na
konfiguráciu sú zadávané takýmto spôsobom:

\begin{center}
\textbf{Príkaz na sťahovanie dát}
\begin{verbatim}
gpsbabel -i skytraq,erase -f /dev/ttyUSB1 -o gpx -F suradnice/ke_1.gpx
\end{verbatim}
Príkazom žiadame zariadenie o dáta ktoré budú zároveň zo zariadenia vymazané.
\paragraph{}
\textbf{Príkaz na nastavenie cieľového bodu}
\begin{verbatim}
gpsbabel -i skytraq,targetlocation=49.56:18.21 -f 
    /dev/ttyUSB 0 -o unicsv -F -  
\end{verbatim}
\textbf{syntax:} gpsbabel = názov programu\\
-i = input - vstupný typ súboru/prístroja\\
-f = file - cesta k súboru\\
-o = output - typ výstupného súboru \\
-F = File názov súboru do ktorého sa dáta zapíšu\\
\end{center}

Je nutné podotknúť že vzorové príkazy boli zamerané na konkrétne prístroje ktoré boli v práci používané.

\subsection{Gebabbel}
Tento program je vytvorený ako nadstavba na program GPSbabel. Hlavnými črtami tohto programu je jednoduchosť a jednoúčelnosť. Je vhodným nástrojom na sťahovanie dát a komunikáciu s prístrojmi. Výhodou tohto programu je výber prístrojov, dát a akcií, ktoré sa majú vykonať. Program disponuje aktuálnou databázou prístrojov a formátov, ktoré GPSbabel podporuje.
