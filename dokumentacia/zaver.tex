%\addcontentsline{toc}{}{}
\addcontentsline{toc}{chapter}{Záver}
\chapter*{Záver}
\paragraph{}
Úlohou v tejto práci bolo vytvorenie aplikácie ktorá bude schopná zobrazovať GPS dáta a komunikovať prístrojmi. Požiadavky na aplikáciu v jej prvej fáze môžme považovať za splnené. 
Ako ďalšie časy ktoré boli do aplikácie pridané sú multijazyčný preklad, ktorý je vhodný hlavne pri zdieľaní takejto aplikácie na internete. Aplikácia bude voľne dostupná, teda je možné ju využívať na študijné ale aj praktické účely pri zobrazovaní trás a aktuálnej pozície. 

Vo svojej prvej fáze je možné aplikáciu spúšťať na rôznych linuxových operačných systémoch v jej plnej funkčnosti. Aplikácia nie je náročná a je možné ju spúšťať aj pri menej výkonnom hardvéri. Samotné vykresľovanie trás je už v aktuálnej verzii plne funkčné aj pri iných platformách. 
\paragraph{}
V tejto práci sme si ozrejmili prácu s Qt knižnicou. Veľký prínos mala aj pre samotné geografické poznatky keďže sme sa často zaoberali výpočtami spoločnými s geografiou. 

Pri porovnávaní formátov sa nám ako najvhodnejší javí formát GPX a aj preto bola práve pre tento typ súboru podporovaná vizualizácia. Samotné programy ako gpsbabel majú podporu aj grafického rozhrania ktoré sa podarilo implementovať do samotnej aplikácie. Prevod dokáže hravo zvládnuť aj menej znalý používateľ tohto programu. 
\paragraph{}
Do budúcnosti je možné aplikáciu rozšíriť o ďalšie moduly, ktoré ju budú robiť komplexnejšou a robustnejšou.
