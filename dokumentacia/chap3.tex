\chapter{GPS prístroje použité v práci}
\paragraph{}
V GPS prístrojoch nastal široký rozmach práve v dnešných časoch. Máme dostupnú
širokú škálu prijímačov z rôznymi presnosťami a to od 10 metrov až po niekoľko
milimetrov. V prístrojoch je snaha o čo najpresnejšie dáta, ale z čo možno najnižšou spotrebou energie pre zaistenie čo najdlhšieho času snímania dát. V tejto práci sme mali k dispozícii dva prístroje ktoré boli od seba značne
odlišné.

Prvý prístroj mal v sebe zabudovanú vnútornú pamäť a dokázal zapisovať prejdené dáta, zobrazovať aktuálnu pozíciu čo ho robilo zaujímavejším a komplexnejším.
Druhý prístroj bol modul na posielanie aktuálnych GPS dát. Výhodou tohto prístroja bol kvalitný signál a tým aj väčšia presnosť.

\subsection*{Spôsob pripojenia}
\paragraph{}
Základným a najbežnejším spôsobom pripojenia GPS prijímača k počítaču je USB\footnote{\textbf{USB} - Univerzálna sériová zbernica (Universal serial bus)}
pripojenie. Mnoho prístrojov disponuje taktiež pripojením pomocou Bluetooth\footnote{\textbf{Bluetooth} - bezdrôtový prenos dát} prenosu.

Dôležité je čo počas spojenia prenášame. Ak je prijímač schopný ukladať dáta je
potrebné nastaviť rôzne atribúty. Ak toho schopný nie je a má možnosť len zistiť
aktuálnu pozíciu túto pozíciu posiela do počítača a je na nás ako dáta spracujeme. 

%\section{Prístroje použité v tejto práci}
%\paragraph{}
\section{GPS multi tracker}
\paragraph{}
Prístroj má rôzne vlastnosti. Je schopný dáta nielen posielať ale ich aj ukladať
do svojej pamäte. Prístroj disponuje čipom Venus 6, ktorého výhodou oproti predošlej verzii je hlavne to, že sa vyznačuje lepším výkonom čím zvládne aj horšie podmienky(zlé počasie, miesta medzi budovami a pod.), a zároveň znížením spotreby čoho výsledkom je dlhšia pracovná doba pri rovnakej kapacite napájacej batérie. Tieto dáta sú rozdelené na trasy ktoré sme ukladali a významné body ktoré je možné osobitne značiť. Ďalšou vlastnosťou je navigácia. 
\subsection{Konfigurácia}
\paragraph{}
Na konfiguráciu existuje niekoľko možností. V OS Linux je dostupný program
skytraq-datalogger, v ktorom je možné nakonfigurovať všetky dôležité vlastnosti pre ukladanie. Nastaviť môžme vlastnosti pripojenia a taktiež spôsob kedy má hardware zapisovať dáta do pamäte.\\\\
\textbf{Pri zápise je možné nastaviť: } \\
- ako rýchlo sa máme pohybovať aby systém zapisoval (minimum a maximum)\\
- aká má byť najmenšia vzdialenosť medzi dvoma bodmi\\
- aký najmenší čas má byť medzi dvoma bodmi
\paragraph{}V prístroji je možné nastaviť čo má robiť ak sa mu zaplní vnútorná pamäť. Máme možnosť prestať zapisovať, alebo zapisovať do pamäte opäť od začiatku.

\subsubsection{Možnosti pripojenia}
\paragraph{}
Tento prístroj je možné pripojiť k počítač dvoma spôsobmi: Cez USB port a cez
bezdrôtové pripojenie Bluetooth.
V Linuxe zvyčjane pripájame prístroj pod portom \textit{ttyUSB0} v adresári dev:  \textit{/dev/ttyUSB0}

\subsection{Sťahovanie dát a zaujímavých bodov}
\paragraph{}
Dáta je možné pomocou programov (spomínaných v druhej kapitole) stiahnuť a následne zobraziť v
programe ktorý podporuje formát GPX. Formát GPX je detailne popísaný v nasledujúcej kapitole.

\subsection{Navigácia}
\paragraph{}
Systém naviguje jednoduchým spôsobom. Po zapísaní cieľovej súradnice do
prístroja je prístroj schopný navigovať k cieľovému bodu na pozícii LF(location
finder) a systém nás naviguje pomocou ôsmich diód ktoré ukazujú smer. 

Keď sa nám podarí priblížiť do vzdialenosti menšej ako 15 metrov tak pri
indikátore Bluetooth pripojenia začne blikať červená dióda.

\section{Navilock GPS Bluetooth Receiver BT-308}
\paragraph{}
Prístroj disponuje čipom SiRF StarII ktorý je schopný prevádzky v predtým
nemysliteľných podmienkach - hustých lesoch, úzkych uliciach a údoliach, krytých
parkoviskách a čiastočne aj v budovách.
Občas sa môže pri týchto procesoroch prejaviť dlhší čas spracovania dát a to
hlavne pri zmene rýchlosti ako pri jazde autom, na bicykli či chôdzou.
Prístroj má iba pripojenie cez Bluetooth. 

\subsection{Komunikácia}
\paragraph{}
Na komunikáciu je potrebné nastaviť prenosovú rýchlosť na 38600 b/s. Na pripojenie k prístroju je potrebné zadať pin kód prístroja, ktorý je nastavený od výrobcu na číslo: 2003.

\subsection{Prijímanie aktuálnej pozície}
\paragraph{}
Prístroj pošle ako GPS dáta v protokole NMEA 0183 zapísané informácie o počte
družíc, pozíciu, celkovú kvalitu signálu atď. Viac o protokole NMEA v 2. Kapitole
