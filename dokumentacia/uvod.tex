\chapter*{ÚVOD}
\addcontentsline{toc}{chapter}{Úvod}
%V úvode autor stručne a výstižne charakterizuje stav poznania alebo praxe v oblasti, ktorá je predmetom záverečnej práce a oboznamuje čitateľa s významom, cieľmi a zámermi práce. Autor v úvode zdôrazňuje, prečo je práca dôležitá a prečo sa rozhodol spracovať danú tému. 
%\section{GPS prístroje}
\paragraph{} 
Prácou s mapami sa zaoberali ľudia už v staroveku, z dôvodu zachytenia prvkov nachádzajúcich sa v okolí. Jeden z dôvodov bol, že mapa nás priviedla vždy do cieľa, na miesto kde sme sa chceli dopraviť. Mapy sa využívali a využívajú všade okolo, môžeme to vidieť v histórii ale aj v aktuálnej dobe.

V súčasnosti je snaha zachytiť každučký detail na mape, vecí a okolnosti ktoré nás zaujímajú, ktoré sú pre nás podstatné. K tomu aby sme mohli zaznamenať na mape rôzne detaily, nás rapídne posúva vpred elektronika z takzvaným GPS systémom. Za posledné dva desaťročia sa trh s touto elektronikou rapídne rozšíril, no nebolo tomu vždy tak. 

GPS je skratka pre Globálny pozičný systém, ktorý vyvinula Americká vláda počas studenej vojny. V časoch studenej vojny mal tento systém úplne iný zmysel ako dnes, jeho vývoj napredoval hlavne ako konkurencia Ruskej vláde a jej vývoju. Časté názory na vývoj GPS sú práve tie že s týmto systémom začala USA a je to aj pravda, ale prvý rádiofrekvenčný signál ktorý tento vývoj podnietil prišiel práve z ruskej družice a tak istý podiel na tomto systéme má určite aj Rusko. Americká vláda s vývojom systému napredovala vred. Dnes je tento navigačný systém jediný na trhu dostupný pre verejnosť, ale je všeobecne známe že funkčných pozičných systémov je omnoho viac, ktoré ale pre verejnosť dostupné nie sú. 

Tento navigačný systém je spustený od roku 1960 a od vtedy prešiel dlhým vývojom, až po dnešnú podobu. Verejnosti bol sprístupňovaný od roku 1990 do roku 1993 po tomto roku už bol plne prístupný. Americká vláda mala obavy o zneužitie systému a tak až do roku 2000 bola v navigačnom systéme zavedená funkcia selektívnej dostupnosti\footnote{\textbf{selektívna dostupnosť} - "Zámerné znepresnenie polohy pre civilných užívateľov GPS, selektívna dostupnosť ang. Selective Availability (SA)" [10]}, ktorá degenerovala presnosť na +-200 metrov. Po zrušení funkcie je dnešná presnosť týchto hodnôt na niekoľko metrov a v špeciálnych geodetických prístrojoch dokonca niekoľko centimetrov. 

Vzhľadom na dostupnosť signálu sa prudko rozšíril aj trh s príjmačmi, následkom čoho bol dopyt po softvéroch ktoré by boli schopné tieto dáta zobrazovať. Bežnou požiadavkou používateľov bolo aby bol systém schopný ukladať body alebo trasy, aby ich bolo možné neskôr nájsť. Na to aby sme to dokázali nám vystačí aj lacnejší a tým aj dostupnejší prístroj pre širokú verejnosť, ktorá ho môže využiť na navigovanie, alebo ukladanie dát.

Podnet na výber tejto témy bol hlavne z tohto dôvodu: Prístrojov na navigáciu je široké spektrum, ale naozaj kvalitný softvér pre široké spektrum užívateľov chýba. Cieľ tejto práce ale nebol vyvinúť softvér so všetkými funkciami o ktoré používateľ má záujem, pretože by nato nebol dostatok času počas jednej práce. Snahou bolo vytvoriť základ pre ďalší, kompletnejší vývoj softvéru.
Ďalším dôvodom pre spracovanie danej témy bolo vytvoriť si rozhľad medzi už dostupným softvérom\footnote{\textbf{softvér} - je označenie pre programové vybavenie počítača} a hardvérom\footnote{\textbf{hardvér} - tu patria všetky počítače a ich súčasti, periférie (zobrazovacie jednotky, zariadenia na vstup a výstup údajov)} GPS.

\section*{Výber softvéru}
\paragraph{}
Pri výbere softvéru sa často stretávame s problémom nekompatibility s
operačným systémom. Dnes máme dostupné rôzne navigačné aplikácie pod OS\footnote{\textbf{OS} - Operačný systém} Windows
ktoré ale nie je možné používať pod OS Linux, alebo inými. Tento problém nám môže vyriešiť
správna voľba programovacieho jazyka a vývojového prostredia. 
V tejto práci sme sa zameriavali na OS Linux, no pri výbere programovacieho
jazyka, ako aj vývojového prostredia sme kládli dôraz na kompatibilitu s inými
OS aby pri budúcom pokračovaní na tomto projekte ho bolo možné zdieľať aplikáciu širokou verejnosťou. 
